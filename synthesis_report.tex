\documentclass[11pt,a4paper]{article}
\usepackage[utf8]{inputenc}
\usepackage[english]{babel}
\usepackage{graphicx}
\usepackage{geometry}
\geometry{margin=2.5cm}
\usepackage{hyperref}
\usepackage{xcolor}
\usepackage{float}
\usepackage{amsmath}

\title{\textbf{Environmental Dashboard for Waste Management} \\ 
\large Comparative Analysis Europe-Africa (1990-2021) \\
\vspace{0.3cm}
\normalsize Synthesis Report}
\author{Bellatreche Mohamed Amine \\ Cherif Ghizlane Imane \\
\vspace{0.3cm}
\small University of Science and Technology of Oran (USTO-MB) \\
\small Department of Computer Science \\
\small State Engineer Diploma in Data Science \\
\vspace{0.2cm}
\small Course: Data Visualization \\
\small Professor: Guerroudji F.}
\date{November 2025}

\begin{document}

\maketitle

\newpage
\tableofcontents
\newpage

\section{Context and Environmental Issues}

We're facing a growing problem: global waste is expected to jump by 70\% before 2050. That's not just an environmental concern---it affects public health and economic stability too. For this project, we analyzed \textbf{49 countries} across Europe (27) and Africa (22) over three decades (1990-2021) to understand how their waste systems compare and identify where intervention is needed most urgently.

	extbf{Interactive Dashboard:} \href{https://eu-africa-waste-comparison-and-forecasting-dashboard.streamlit.app/}{eu-africa-waste-comparison-and-forecasting-dashboard.streamlit.app}

\textbf{European countries (27):} Austria, Belgium, Czechia, Denmark, Estonia, Finland, France, Germany, Greece, Hungary, Iceland, Ireland, Italy, Latvia, Lithuania, Luxembourg, Netherlands, Norway, Poland, Portugal, Slovakia, Slovenia, Spain, Sweden, Switzerland, Turkey, United Kingdom

\textbf{African countries (22):} Algeria, Benin, Botswana, Burkina Faso, Burundi, Cape Verde, Egypt, Ghana, Guinea, Kenya, Lesotho, Madagascar, Mauritius, Morocco, Niger, South Africa, Sudan, Tanzania, Togo, Tunisia, Zambia, Zimbabwe

\subsection{Regional Challenges}

\textbf{Europe:} Our 27 European countries show recycling rates from 11\% to 48\% as of 2015:
\begin{itemize}
    \item Germany leads at 47.8\% recycling (2015)
    \item Top performers also include Austria, Belgium, Netherlands, Sweden (all above 35\%)
    \item Mid-range: France (22.3\%), Italy (29.1\%), Spain (16.8\%)
    \item Lower performers: Turkey (11\%), Greece, Poland (all below 20\%)
    \item Average across 27 countries: approximately 30\%
\end{itemize}

\textbf{Africa:} Our 22 African countries span both North and Sub-Saharan regions:
\begin{itemize}
    \item \textbf{North Africa:} Algeria, Egypt, Morocco, Tunisia
    \item \textbf{Sub-Saharan:} Benin, Botswana, Burkina Faso, Burundi, Cape Verde, Ghana, Guinea, Kenya, Lesotho, Madagascar, Mauritius, Niger, South Africa, Sudan, Tanzania, Togo, Zambia, Zimbabwe
    \item Algeria: 254 kg/capita (2021), most complete time series
    \item No official recycling data available for any African country in dataset
    \item Wide geographic representation across the continent
\end{itemize}

\textbf{Our research objectives:} We wanted to understand exactly how these waste systems differ structurally. Can machine learning help us predict future trends for better planning? And critically, which countries need priority intervention right now?

\section{Dataset Description}

\subsection{Data Sources}

\textbf{1. UN Environment Waste Generation} (2000-2021): From this dataset, we extracted 49 countries with sufficient data: 27 European and 22 African countries. Data breaks down waste by sector where available. Many African countries had gaps requiring interpolation (Algeria reported in 2002, 2005, 2009, then continuously from 2014-2021). European countries generally have more complete time series. \url{https://ourworldindata.org/grapher/total-waste-generation}

\textbf{2. OECD Recycling Statistics} (1990-2015): From this dataset, we used 27 European countries with time series data ranging from 1990-2015. Countries include Austria, Belgium, Czechia, Denmark, Estonia, Finland, France, Germany, Greece, Hungary, Iceland, Ireland, Italy, Latvia, Lithuania, Luxembourg, Netherlands, Norway, Poland, Portugal, Slovakia, Slovenia, Spain, Sweden, Switzerland, Turkey, and United Kingdom. Africa completely lacks formal recycling data. \url{https://ourworldindata.org/grapher/recycling-rates-paper-and-cardboard}

\textbf{3. World Bank Population} (2020 static): Population data for all 49 countries used for per capita waste calculations where data completeness allows.

\subsection{Preprocessing Pipeline}

\textbf{1. Temporal Interpolation:} Since the data was collected biennially (every two years), we applied linear interpolation to fill in the missing years. We only did this when we had at least 2 actual data points per country to work with---otherwise the interpolation wouldn't be reliable.

\textbf{2. Per Capita Normalization:} To make fair comparisons, we divided total waste by population to get per-person figures. This was important because it lets us compare countries like Germany (520 kg per person) with Luxembourg (630 kg per person) meaningfully, even though Germany's total volume is roughly 100 times higher.

\textbf{3. Dataset Merging:} Outer join on [country, year]. Europe (4 countries) 1990-2015: recycling + waste; 2016-2021: waste only. Africa (4 countries) 2000-2021: waste only.

\textbf{Data Limitations:} Working with this data had its challenges. Recycling statistics stop at 2015---that's when OECD published their last comprehensive report. Africa doesn't have any formal recycling data. The dataset is limited to just 8 countries total, with no representation from Eastern Europe, Sub-Saharan Africa, or other regions. Egypt and Tunisia have very sparse data points requiring heavy interpolation.

\begin{table}[h]
\centering
\small
\begin{tabular}{|l|c|c|}
\hline
\textbf{Metric} & \textbf{Europe} & \textbf{Africa} \\
\hline
Countries & 27 & 22 \\
Recycling data & 1990-2015 (all) & None available \\
Waste data & 2000-2021 & 2000-2021 \\
After interpolation & 1990-2021 (annual) & 2000-2021 (annual) \\
Data completeness & Generally high & Varies (sparse for many) \\
\hline
\end{tabular}
\caption{Dataset characteristics - 49 countries analyzed}
\end{table}

\section{Indicator Selection and Justification}

\textbf{1. Recycling Rate (\%, Europe only):} Based on OECD data ending in 2015. Our 27 European countries show a range from approximately 11\% (Turkey) to 47.8\% (Germany), with an average of around 30\%. Top performers include Germany (47.8\%), Austria, Belgium, Netherlands, and Sweden (all above 35\%). Lower performers include Turkey, Greece, and several Eastern European countries below 20\%.

\textbf{2. Waste Per Capita (kg/person/year):} We calculated this metric across all 49 countries where population data was available. European countries generally show higher per capita waste (400-600 kg/year), while African countries vary widely (150-400 kg/year). Algeria shows 254 kg/capita (2021) as a North African example.

\textbf{3. Environmental Risk Score (0-100):} We created a scoring system that considers recycling efficiency, waste generation levels, and growth rates across all 49 countries. The 27 European countries with recycling data have lower baseline risk scores, while 22 African countries without recycling infrastructure start with higher baseline penalties. The color coding makes it easy to spot problems: green means low risk (<30), yellow is moderate (30-60), and red means urgent intervention needed (>60).

\textbf{4. ARIMA Forecasts (2022-2026):} This is a machine learning model that learns from actual waste patterns to predict the next 5 years. With our dataset of 49 countries, ARIMA performance varies---countries with complete time series (like most Western European countries) produce reliable forecasts (green badges), while countries with sparse or interrupted data fall back to simpler linear predictions (orange badges). On the charts, you'll see dotted lines extending the solid historical data into the future.

\section{Dashboard Visualizations: Design Rationale}

We designed an interactive dashboard with multiple visualizations across several pages. Each chart type was chosen deliberately for specific analytical purposes, following established best practices in data visualization. Here's our reasoning:

\subsection{Why These Chart Types?}

\textbf{1. Choropleth Maps (Geographic Pages):}
\begin{itemize}
    \item \textbf{Purpose:} Display spatial patterns immediately
    \item \textbf{Why this works:} Maps leverage our natural ability to process geographic shapes much faster than scanning tables of numbers
    \item \textbf{Color choice:} We used Red-Yellow-Green (RdYlGn) for recycling rates where red signals problems and green shows success. For waste generation, we used a Reds scale where darker shades indicate more serious issues
    \item \textbf{Interactive tooltips:} Hovering over any country reveals exact numbers without cluttering the visual
\end{itemize}

\textbf{2. Horizontal Bar Charts (Rankings):}
\begin{itemize}
    \item \textbf{Purpose:} Enable direct country-to-country comparison among the 4 European countries
    \item \textbf{Why this works:} Country names are much easier to read when placed on the Y-axis rather than rotated at awkward angles below bars
    \item \textbf{Sorted by value:} We always sort descending---top performers at the top, bottom performers at the bottom. This makes the ranking obvious
    \item \textbf{Color gradient:} Colors intensify with values (e.g., darker red = higher risk level)
\end{itemize}

\textbf{3. Line Charts (Temporal Trends):}
\begin{itemize}
    \item \textbf{Purpose:} Track changes over 32 years (1990-2021)
    \item \textbf{Why this works:} Lines naturally convey continuity and direction
    \item \textbf{Dotted forecasts:} Visual distinction between historical (solid) and predicted (dashed)
    \item \textbf{Markers:} Dots show actual data points vs. interpolated values
\end{itemize}

\textbf{4. Stacked Area Charts (Sector Analysis):}
\begin{itemize}
    \item \textbf{Purpose:} Show part-to-whole relationships over time
    \item \textbf{Why this works:} Band width = sector contribution, total height = aggregate
    \item \textbf{Color semantics:} Red (households=primary waste), Orange (construction=second largest), Blue (manufacturing=industrial), Purple (services=tertiary sector)
    \item \textbf{Maximum contrast:} Colors chosen from different parts of the color wheel for clear visual distinction
    \item \textbf{Layer order:} Largest sectors at bottom for stability
\end{itemize}

\textbf{5. Scatter Plots (Performance Analysis):}
\begin{itemize}
    \item \textbf{Purpose:} Explore 2-variable relationships (recycling vs waste) for the 4 European countries with both metrics
    \item \textbf{Quadrants:} Dashed lines divide into categories
    \item \textbf{Size encoding:} Bubble size = total waste volume (3rd dimension)
    \item \textbf{Color by category:} Visual grouping of similar performers
\end{itemize}

\textbf{6. Heatmaps (Correlation Matrix):}
\begin{itemize}
    \item \textbf{Purpose:} Identify countries with similar recycling patterns (4 European countries)
    \item \textbf{Why this works:} Grid format shows all pairwise relationships
    \item \textbf{Diverging scale:} Blue=positive correlation (similar trends), Red=negative (opposite)
    \item \textbf{Practical use:} With only 4 countries, patterns are limited but can identify best-practice sharing opportunities
\end{itemize}

\textbf{7. KPI Cards with Gradients:}
\begin{itemize}
    \item \textbf{Purpose:} Highlight key metrics at dashboard entry
    \item \textbf{Visual hierarchy:} Large numbers (3rem font), small labels (descriptive text)
    \item \textbf{Color psychology:} Green (recycling=positive), Red (waste=problem), Purple (champions=excellence), Blue (targets=goals)
    \item \textbf{Gradient backgrounds:} Aesthetic appeal while maintaining contrast for accessibility
\end{itemize}

\subsection{Color Psychology Strategy}

\textbf{Our semantic color mapping:}
\begin{itemize}
    \item \textbf{Green shades:} We used these for recycling rates and positive environmental actions. Green has natural associations with nature, growth, and positivity
    \item \textbf{Red shades:} Reserved for waste generation, environmental risks, and household waste (primary source). Red universally signals danger and urgency
    \item \textbf{Orange shades:} Used for construction waste and moderate concerns. Orange signals caution while remaining visually distinct from red
    \item \textbf{Blue shades:} Applied to manufacturing/industrial waste, targets, and goals. Blue provides cool-tone contrast against warm colors and conveys trust and institutional stability
    \item \textbf{Purple shades:} Designated for service sector waste and top performers/champions. Purple culturally suggests prestige and is easily distinguishable from other colors
\end{itemize}

\textbf{Why we think these colors work well:}
\begin{itemize}
    \item \textbf{Intuitive understanding:} Users grasp red=bad and green=good immediately without needing to consult legends
    \item \textbf{Maximum visual separation:} Stacked area charts use colors from different color wheel segments (red-orange-blue-purple) ensuring each sector is clearly distinguishable even when bands are narrow
    \item \textbf{Accessibility:} The diverging scales we chose (RdBu, RdYlGn) remain distinguishable for people with common color vision deficiencies. Sector colors avoid red-green adjacency that would confuse colorblind users
    \item \textbf{High contrast:} We ensured all color combinations meet WCAG AA standards for readability
    \item \textbf{Consistency:} Each color maintains the same meaning throughout all visualizations
\end{itemize}

\subsection{Design Principles Applied}

\textbf{1. Tufte's Data-Ink Ratio:} Maximize information, minimize decoration. No 3D effects, minimal gridlines, removed chartjunk.

\textbf{2. Cleveland \& McGill Hierarchy:} Position > Length > Angle > Area > Color. Used bar charts (length) over pie charts (angle) for accuracy.

\textbf{3. Progressive Disclosure:} Simple overview page → detailed analytics. Hover tooltips show extra data on demand.

\textbf{4. Responsive Sizing:} Chart height = $\max(400px, n_{countries} \times 30px)$ ensures readability even with 8 countries (though this is less critical with smaller datasets).

\textbf{5. Storytelling Flow:} Page order guides users: Overview → Geographic → Analytics → Predictions → Risks (general to specific).

\begin{figure}[H]
\centering
\includegraphics[width=0.90\textwidth]{screenshots/01_overview_europe.png}
\caption{Overview Page - Europe: KPI cards display recycling rate, waste per capita, and risk score. Temporal line chart shows 32-year trend (1990-2021) with smoothing spline. Sector breakdown via stacked area chart reveals household waste dominance.}
\end{figure}

\begin{figure}[H]
\centering
\includegraphics[width=0.90\textwidth]{screenshots/02_overview_comparison.png}
\caption{Overview Comparison: Side-by-side metrics highlight differences between 4 European countries (France, Germany, Italy, Spain) and 4 African countries (Algeria, Egypt, Morocco, Tunisia). Bar charts compare absolute volumes and rates where per capita data is available (limited to Algeria for Africa).}
\end{figure}

\begin{figure}[H]
\centering
\includegraphics[width=0.90\textwidth]{screenshots/03_geographic_europe.png}
\caption{Geographic Analysis - Europe: Choropleth map uses diverging color scale (ColorBrewer RdYlGn) to visualize recycling rates across 4 countries. Germany (47.8\%) leads, while Spain lags (16.8\%). France (22.3\%) and Italy (29.1\%) fall in the middle range. Interactive tooltips provide country-specific data.}
\end{figure}

\begin{figure}[H]
\centering
\includegraphics[width=0.90\textwidth]{screenshots/04_geographic_africa.png}
\caption{Geographic Analysis - Africa: Map displays waste generation for 4 North African countries (Algeria, Egypt, Morocco, Tunisia). Algeria shows 254 kg/capita (2021). Other countries lack complete per capita calculations. This map only covers North Africa, with no Sub-Saharan representation.}
\end{figure}

\begin{figure}[H]
\centering
\includegraphics[width=0.90\textwidth]{screenshots/06_advanced_analytics.png}
\caption{Advanced Analytics: Scatter plot matrix examines correlations. Strong positive relationship between GDP/capita and waste/capita (r=0.58) confirms consumption-waste linkage. Heatmap below quantifies correlation coefficients for all indicator pairs.}
\end{figure}

\begin{figure}[H]
\centering
\includegraphics[width=0.90\textwidth]{screenshots/08_predictions_europe.png}
\caption{Predictions - Europe: ARIMA forecasts (2022-2026) shown as dotted lines extending historical data. Model usage indicators show which countries successfully used ARIMA (green badges) vs. linear regression fallback (orange badges). Confidence intervals (shaded area) widen for longer horizons.}
\end{figure}

\begin{figure}[H]
\centering
\includegraphics[width=0.90\textwidth]{screenshots/09_predictions_africa.png}
\caption{Predictions - Africa: Forecasts for 4 North African countries (Algeria, Egypt, Morocco, Tunisia). Algeria has the most reliable forecast due to more complete data. Egypt and Tunisia forecasts rely more heavily on interpolation and linear methods due to sparse historical data (2000-2021).}
\end{figure}

\begin{figure}[H]
\centering
\includegraphics[width=0.90\textwidth]{screenshots/11_risk_assessment_comparison.png}
\caption{Risk Assessment Comparison: Horizontal bar chart ranks countries by risk score. With only 8 countries total (4 Europe, 4 Africa), the comparison is limited but demonstrates the methodology. European countries benefit from recycling data (lower risk scores), while African countries are penalized for lack of formal recycling systems (higher risk scores). Color coding (green/yellow/red) enables quick identification of relative risk levels.}
\end{figure}

\section{Results Interpretation}

\subsection{Europe: The Optimization Challenge}

\textbf{Current state (2015 recycling data):} Our 27 European countries achieve an average of approximately 30\% recycling, with significant variation (11\%-48\% range). 

\textbf{Leading countries:} Germany tops the list at 47.8\% (2015), showing steady improvement from 23.3\% in 1993. Other top performers include Austria, Belgium, Netherlands, and Sweden (all above 35\%). These countries implemented comprehensive systems early---deposit-return schemes, pay-as-you-throw pricing, and Extended Producer Responsibility policies.

\textbf{Countries falling behind:} Turkey sits at approximately 11\%, Greece and several Eastern European countries below 20\%. Spain at 16.8\% and France at 22.3\% are below average. Italy performs reasonably well at 29.1\%.

\textbf{Future projections (2022-2026):} Our ARIMA models forecast waste trends for the 27 European countries. Most Western European countries have stable time series allowing reliable projections. Eastern European countries and Turkey show more variability requiring linear fallback methods. Note that recycling data only extends to 2015, so post-2015 recycling projections are unavailable.

\subsection{Africa: The Infrastructure Crisis}

\textbf{The current situation (2021):} Our 22 African countries span North and Sub-Saharan regions, with no official recycling data for any country.

\textbf{Data coverage:} Algeria provides the most complete dataset with 254 kg/capita waste generation (2021) and continuous reporting from 2014-2021. Other countries have varying levels of data completeness. North African countries (Algeria, Egypt, Morocco, Tunisia) generally have better data than many Sub-Saharan countries.

\textbf{What's coming (2022-2026):} ARIMA forecasts are challenging with sparse African data. Countries with more complete datasets (like Algeria and South Africa) allow for better projections, while countries with limited historical data require linear fallback methods.

\textbf{Geographic diversity:} The dataset includes 22 countries representing diverse regions: North Africa (4 countries), West Africa (5 countries), East Africa (5 countries), Central Africa (2 countries), and Southern Africa (6 countries), providing continental-scale insights.

\subsection{Key Correlations}

\textbf{Observed relationships:} With 49 countries, we can observe meaningful patterns. Among the 27 European countries with recycling data, there's clear evidence that higher recycling rates correlate with lower environmental risk scores. Germany (47.8\% recycling) demonstrates best-in-class performance, while countries below 20\% show higher risk indicators.

\textbf{Germany's leadership:} Germany's waste data shows relatively stable generation levels from 2004-2020 (ranging 310-338 million tonnes), demonstrating that mature economies can maintain waste levels while continuing economic activity.

\textbf{Regional patterns:} Western European countries generally show higher recycling rates (30-48\%) compared to Eastern European countries and Turkey (11-25\%). Sub-Saharan African countries show lower per capita waste generation than North African countries, likely reflecting different levels of urbanization and consumption patterns.

\section{Recommendations and Action Plans}

\subsection{Europe: Accelerating the Circular Economy}

\textbf{What needs to happen soon (2025-2027):}
\begin{enumerate}
    \item Lower-performing European countries (Turkey, Greece, Eastern Europe) need to adopt best practices from Germany, Austria, and Netherlands
    \item Extend recycling data collection beyond 2015 to track post-2015 progress across all 27 European countries
    \item Improve data completeness and consistency across all 22 African countries
    \item Establish baseline recycling measurement systems in at least 5-10 African countries as pilot programs
\end{enumerate}

\textbf{Looking further ahead (2028-2035):}
\begin{enumerate}
    \item Expand European recycling best practices across all 27 countries, targeting 40\% minimum for all by 2030
    \item Establish recycling infrastructure in priority African countries (starting with those with better data: Algeria, South Africa, Egypt)
    \item Create cross-regional learning networks between high-performing European countries and African countries
    \item Standardize waste measurement methodology across all 49 countries for better comparability
\end{enumerate}

\subsection{Africa: Building the Foundation}

\textbf{Urgent priorities (2025-2028):}
\begin{enumerate}
    \item \textbf{Improve data collection:} Priority focus on countries with incomplete time series (Kenya, Ghana, Tanzania, Sudan, and others)
    \item \textbf{Establish recycling baseline:} All 22 African countries need initial recycling measurement systems to enable progress tracking
    \item \textbf{Regional hubs:} Establish 3-4 regional waste management centers (North Africa, West Africa, East Africa, Southern Africa)
    \item \textbf{Capacity building:} Leverage countries with better data (Algeria, South Africa) as regional training centers
\end{enumerate}

\textbf{Long-term strategy (2029-2035):}
\begin{enumerate}
    \item Build on successful countries: Replicate Algeria and South Africa's data collection methods across all 22 countries
    \item Foster Europe-Africa partnerships: Top European performers (Germany, Austria, Belgium) mentor African countries through city-to-city programs
    \item Regional integration: Create waste management zones covering multiple countries for economies of scale (e.g., West Africa zone: Ghana, Benin, Togo, Burkina Faso)
\end{enumerate}

\subsection{Using Analytics for Better Decisions}

\textbf{How ARIMA helps:} For countries with complete data (like Germany and Algeria), ARIMA provides 5-year forecasts to plan infrastructure capacity. For countries with sparse data (Egypt, Tunisia), the linear fallback at least indicates general trends.

\textbf{How risk scoring helps:} The scoring system highlights which countries need priority attention, though with only 8 countries the comparative analysis is limited.

\textbf{The dashboard as a practical tool:} Demonstrates waste management visualization and forecasting methodology that could be applied to expanded datasets with more countries and complete time series data.

\section*{Conclusion}

Our analysis of 49 countries (27 European, 22 African) provides comprehensive insights into waste management across both continents. 

\textbf{Main takeaways:} Germany leads European recycling at 47.8\% (2015), while Turkey lags at 11\%---a 37-point gap demonstrates massive variability even within Europe. All 22 African countries completely lack formal recycling measurement, preventing comparative analysis. Algeria provides the most complete African waste time series (254 kg/capita in 2021). The dataset includes good geographic diversity: all major European regions represented, and African countries spanning North, West, East, Central, and Southern regions.

\textbf{Data quality achievements:} With 49 countries, we have sufficient sample size for meaningful continental comparisons. The 27 European countries with recycling data enable analysis of what works and what doesn't. The 22 African countries provide continent-wide waste generation baseline.

\textbf{What this means going forward:} This dashboard demonstrates that comprehensive multi-country analysis is achievable and valuable. The 49-country dataset enables: (1) identification of regional best practices, (2) cross-country learning opportunities, (3) baseline measurement for SDG 12.5 tracking, and (4) forecasting models that can guide infrastructure planning.

The dashboard we built provides a working framework that brings together machine learning forecasts (ARIMA), rule-based risk assessment, and intuitive visualizations for 49 countries across two continents.

\section*{References}

\begin{enumerate}
    \item Our World in Data (2023). \textit{Total Waste Generation by Country}. Available at: \url{https://ourworldindata.org/grapher/total-waste-generation}
    
    \item Our World in Data (2023). \textit{Municipal Waste Recycling Rates}. Available at: \url{https://ourworldindata.org/grapher/recycling-rates-paper-and-cardboard}
    
    \item World Bank (2020). \textit{Population Data by Country}. World Bank Open Data Portal.
    
    \item Tufte, E. R. (2001). \textit{The Visual Display of Quantitative Information} (2nd ed.). Graphics Press.
    
    \item Cleveland, W. S., \& McGill, R. (1984). Graphical Perception: Theory, Experimentation, and Application to the Development of Graphical Methods. \textit{Journal of the American Statistical Association}, 79(387), 531-554.
    
    \item European Union (2018). \textit{Waste Framework Directive 2008/98/EC}. Official Journal of the European Union.
    
    \item UNEP (2021). \textit{Africa Waste Management Outlook}. United Nations Environment Programme.
    
    \item Box, G. E. P., Jenkins, G. M., Reinsel, G. C., \& Ljung, G. M. (2015). \textit{Time Series Analysis: Forecasting and Control} (5th ed.). Wiley.
\end{enumerate}

\end{document}
