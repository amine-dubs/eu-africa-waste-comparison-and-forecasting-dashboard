\documentclass[11pt,a4paper]{article}
\usepackage[utf8]{inputenc}
\usepackage[T1]{fontenc}
\usepackage[english]{babel}
\usepackage{graphicx}
\usepackage{hyperref}
\usepackage{geometry}
\usepackage{float}
\usepackage{caption}
\usepackage{subcaption}
\usepackage{xcolor}
\usepackage{listings}
\usepackage{amsmath}
\usepackage{booktabs}
\usepackage{fancyhdr}

\geometry{margin=2.5cm}
\pagestyle{fancy}
\fancyhf{}
\rhead{Environmental Dashboard - Waste Management}
\lhead{Synthesis Report}
\cfoot{\thepage}

\definecolor{darkblue}{RGB}{0,51,102}
\hypersetup{
    colorlinks=true,
    linkcolor=darkblue,
    filecolor=darkblue,
    urlcolor=darkblue,
    citecolor=darkblue
}

\title{\textbf{\Huge Environmental Dashboard}\\
\Large Comparative Analysis of Waste Management\\
Europe \& Africa}
\author{Bellatreche Mohamed Amine \and Cherif Ghizlane Imane}
\date{\today}

\begin{document}

\maketitle
\thispagestyle{empty}

\begin{abstract}
This report presents an interactive environmental dashboard developed to analyze and compare waste management practices between European and African countries. Using data from Our World in Data (\url{https://ourworldindata.org/grapher/total-waste-generation}) and OECD recycling statistics (\url{https://ourworldindata.org/grapher/recycling-rates-paper-and-cardboard}), the dashboard provides insights into recycling rates, waste generation trends, and environmental risks. The platform integrates ARIMA machine learning predictions and rule-based risk assessment to support evidence-based decision-making in environmental policy.
\end{abstract}

\newpage
\tableofcontents
\newpage

\section{Context and Environmental Issues}

\subsection{Global Waste Crisis}
The world is facing an unprecedented waste management crisis. According to the World Bank, global waste generation is expected to increase by 70\% by 2050 if current trends continue. This escalating problem poses severe threats to:

\begin{itemize}
    \item \textbf{Environmental health}: Pollution of air, water, and soil
    \item \textbf{Climate change}: Greenhouse gas emissions from landfills and waste treatment
    \item \textbf{Public health}: Disease transmission and toxic exposure
    \item \textbf{Economic sustainability}: Resource depletion and lost economic opportunities
\end{itemize}

\subsection{North-South Divide}
The disparity in waste management infrastructure and practices between developed (Global North) and developing (Global South) nations is striking:

\begin{itemize}
    \item \textbf{Europe}: Advanced recycling infrastructure, circular economy initiatives, strict environmental regulations (e.g., EU Waste Framework Directive targeting 50\% recycling by 2020)
    \item \textbf{Africa}: Rapidly increasing waste generation, limited recycling infrastructure, informal waste management sectors, and growing environmental challenges
\end{itemize}

\subsection{Project Motivation}
This dashboard was developed to:
\begin{enumerate}
    \item Provide a comprehensive comparative analysis between European and African waste management systems
    \item Identify best practices and areas requiring improvement
    \item Enable data-driven policy decisions through predictive analytics
    \item Support environmental risk assessment and resource allocation
    \item Promote awareness of global waste management disparities
\end{enumerate}

\section{Dataset Description}

\subsection{Data Sources}

\subsubsection{OECD Municipal Waste Recycling Rate}
\begin{itemize}
    \item \textbf{Coverage}: 27 European countries
    \item \textbf{Temporal range}: 1990--2015 (biennial collection, interpolated to annual)
    \item \textbf{Metric}: Percentage of municipal waste recycled
    \item \textbf{Countries}: France, Germany, Italy, Spain, Belgium, Netherlands, Austria, Denmark, Sweden, Finland, Norway, Switzerland, Poland, Portugal, Greece, Ireland, Czechia, United Kingdom, Luxembourg, Slovenia, Slovakia, Estonia, Hungary, Iceland, Latvia, Lithuania, Turkey
\end{itemize}

\subsubsection{UN Environment Total Waste Generation}
\begin{itemize}
    \item \textbf{Coverage}: 100+ countries globally (22 African countries selected)
    \item \textbf{Temporal range}: 2000--2021 (biennial collection, interpolated to annual)
    \item \textbf{Metric}: Total household waste in tonnes
    \item \textbf{African countries}: Algeria, Egypt, Morocco, Tunisia, South Africa, Kenya, Ghana, Botswana, Mauritius, Benin, Burkina Faso, Burundi, Cape Verde, Guinea, Lesotho, Madagascar, Niger, Sudan, Tanzania, Togo, Zambia, Zimbabwe
\end{itemize}

\subsection{Data Preprocessing}

\subsubsection{Interpolation Strategy}
To address missing years in both datasets (data collected biennially):
\begin{itemize}
    \item \textbf{Linear interpolation}: Applied to fill odd years (2011, 2013, 2015, etc.)
    \item \textbf{Recycling rates}: Interpolated for all European countries
    \item \textbf{Waste generation}: Interpolated for both European and African countries
    \item \textbf{Mathematical approach}:
    \begin{equation}
        y_{t} = y_{t-1} + \frac{(y_{t+1} - y_{t-1})}{2}
    \end{equation}
\end{itemize}

\subsubsection{Calculated Indicators}
\begin{itemize}
    \item \textbf{Waste per capita}: 
    \begin{equation}
        W_{pc} = \frac{W_{total} \times 1000}{P \times 10^6} \text{ (kg/person/year)}
    \end{equation}
    where $W_{total}$ is total waste in tonnes, $P$ is population in millions
    
    \item \textbf{Growth rate}: Year-over-year percentage change in waste generation
    
    \item \textbf{Risk scores}: Composite indicators based on multiple factors (detailed in Section 3.3)
\end{itemize}

\subsection{Data Quality Considerations}
\begin{itemize}
    \item \textbf{Limitation}: African countries have no recycling rate data in OECD dataset
    \item \textbf{Solution}: Dashboard provides separate analyses and comparative metrics based on waste generation alone
    \item \textbf{Missing data}: Handled through interpolation for sparse time series (e.g., Kenya with single data point, Benin with three points)
\end{itemize}

\section{Indicator Selection and Justification}

\subsection{Primary Indicators}

\subsubsection{Recycling Rate (\%)}
\textbf{Relevance}: Direct measure of circular economy progress and waste diversion from landfills.

\textbf{Justification}:
\begin{itemize}
    \item Key performance indicator in EU Waste Framework Directive
    \item Reflects infrastructure investment and policy effectiveness
    \item Enables benchmarking against 30\% and 50\% recycling targets
\end{itemize}

\subsubsection{Waste Per Capita (kg/person/year)}
\textbf{Relevance}: Normalizes waste generation by population, enabling fair comparison.

\textbf{Justification}:
\begin{itemize}
    \item Accounts for country size differences
    \item Correlates with economic development and consumption patterns
    \item Universal metric applicable to all countries (Africa \& Europe)
\end{itemize}

\subsubsection{Total Waste Generation (tonnes)}
\textbf{Relevance}: Absolute scale of waste management challenge.

\textbf{Justification}:
\begin{itemize}
    \item Critical for infrastructure planning and resource allocation
    \item Reveals the magnitude of environmental impact
    \item Useful for regional aggregation and comparison
\end{itemize}

\subsection{Derived Indicators}

\subsubsection{Growth Rate (\%/year)}
\textbf{Calculation}:
\begin{equation}
    GR = \frac{W_t - W_{t-1}}{W_{t-1}} \times 100
\end{equation}

\textbf{Justification}:
\begin{itemize}
    \item Identifies accelerating waste generation trends
    \item Early warning indicator for intervention planning
    \item Captures dynamic changes over time
\end{itemize}

\subsection{Risk Assessment Scores}

\subsubsection{European Risk Score (0--100)}
\textbf{Components}:
\begin{align}
    R_{europe} &= w_1 \cdot (100 - R_{rate}) + w_2 \cdot f(W_{pc}) + w_3 \cdot f(GR)\\
    \text{where: } & \notag \\
    w_1 &= 0.40 \text{ (recycling rate weight)} \notag \\
    w_2 &= 0.35 \text{ (waste per capita weight)} \notag \\
    w_3 &= 0.25 \text{ (growth rate weight)} \notag
\end{align}

\textbf{Risk thresholds}:
\begin{itemize}
    \item Low recycling: $R_{rate} < 20\%$ (high risk)
    \item High waste per capita: $W_{pc} > 500$ kg/year
    \item Rapid growth: $GR > 3\%$/year
\end{itemize}

\subsubsection{African Risk Score (0--100)}
\textbf{Components} (no recycling data available):
\begin{align}
    R_{africa} &= 35 + w_1 \cdot f(W_{pc}) + w_2 \cdot f(GR) + w_3 \cdot f(W_{total})\\
    \text{Base risk} &= 35 \text{ (no recycling infrastructure)} \notag
\end{align}

\textbf{Risk thresholds}:
\begin{itemize}
    \item Baseline: 35 points (absence of formal recycling systems)
    \item High waste per capita: $W_{pc} > 400$ kg/year (+25 points)
    \item Rapid growth: $GR > 3\%$/year (+30 points)
    \item High volume: $W_{total} > 10M$ tonnes (+10 points)
\end{itemize}

\subsection{Predictive Analytics and Risk Assessment}

\subsubsection{Waste Forecasting: Machine Learning Model}
\textbf{Algorithm}: ARIMA (Autoregressive Integrated Moving Average) with Linear Regression Fallback

\textbf{Mathematical formulation}:
\begin{equation}
    \hat{W}_{t+k} = \phi_1 W_{t-1} + \phi_2 W_{t-2} + \cdots + \phi_p W_{t-p} + \theta_1 \epsilon_{t-1} + \theta_2 \epsilon_{t-2} + \cdots + \theta_q \epsilon_{t-q}
\end{equation}

where:
\begin{itemize}
    \item $\hat{W}_{t+k}$ = Predicted waste per capita at future time $t+k$
    \item $\phi_i$ = Autoregressive parameters (learn from past waste values)
    \item $W_{t-i}$ = Past waste observations
    \item $\theta_j$ = Moving average parameters (learn from past errors)
    \item $\epsilon_{t-j}$ = Past prediction errors
    \item ARIMA order: $(p=1, d=1, q=1)$ where $p$ = autoregressive order, $d$ = differencing, $q$ = moving average order
\end{itemize}

\textbf{Training Parameters}:
\begin{itemize}
    \item \textbf{Primary algorithm}: statsmodels ARIMA(1,1,1)
    \item \textbf{Fallback algorithm}: scikit-learn LinearRegression (if ARIMA fails)
    \item \textbf{Window size}: User-configurable (3, 5, 7, or 10 years)
    \item \textbf{Forecast horizon}: 5 years ahead
    \item \textbf{Training data}: Uses most recent $n$ years only (rolling window)
    \item \textbf{ARIMA features}: Past waste values (time series patterns)
    \item \textbf{Linear features}: Time/year (if fallback triggered)
    \item \textbf{Target}: Waste per capita (kg/person/year)
\end{itemize}

\textbf{Model Justification}:
\begin{itemize}
    \item \textbf{ARIMA advantages}:
    \begin{itemize}
        \item Uses actual waste values, not just year numbers (learns from patterns like "up 5kg, down 2kg, up 4kg")
        \item Captures non-linear trends and fluctuations better than linear models
        \item Autoregressive component: Leverages correlation between consecutive years
        \item Differencing: Removes trends to focus on changes
        \item Moving average: Accounts for prediction errors to improve accuracy
    \end{itemize}
    \item \textbf{Rolling window}: Captures recent trends more accurately than full historical data
    \item \textbf{Fallback mechanism}: Linear Regression ensures predictions even with sparse data
    \item \textbf{Computational efficiency}: Real-time predictions in interactive dashboard
    \item \textbf{Interpretability}: Model coefficients reveal waste generation dynamics
    \item \textbf{Adaptability}: Smaller windows (3-5 years) respond to recent policy changes
\end{itemize}

\textbf{Model Validation}:
\begin{itemize}
    \item Non-negative predictions enforced ($\hat{W} \geq 0$)
    \item Minimum 3 historical data points required for training
    \item ARIMA stationarity assumptions checked via differencing ($d=1$)
    \item Automatic fallback to Linear Regression if ARIMA convergence fails
    \item Visual indicators show which model was used (ARIMA vs. fallback)
    \item Predictions visualized with confidence context (historical vs. forecast)
\end{itemize}

\subsubsection{Risk Assessment: Rule-Based Expert System}
\textbf{Important Note}: Risk scoring uses a \textbf{rule-based expert system}, not machine learning. This approach was chosen for transparency and domain expert interpretability.

\textbf{European Risk Score Formula}:
\begin{align}
    R_{europe} &= w_1(R_{rate}) + w_2(W_{pc}) + w_3(GR) \notag \\
    &\text{where:} \notag \\
    w_1(R_{rate}) &= \begin{cases}
        40 & \text{if } R_{rate} < 20\% \\
        25 & \text{if } 20\% \leq R_{rate} < 30\% \\
        10 & \text{if } 30\% \leq R_{rate} < 40\% \\
        0 & \text{if } R_{rate} \geq 40\%
    \end{cases} \notag \\
    w_2(W_{pc}) &= \begin{cases}
        30 & \text{if } W_{pc} > 600 \text{ kg/yr} \\
        20 & \text{if } 500 < W_{pc} \leq 600 \\
        10 & \text{if } 400 < W_{pc} \leq 500 \\
        0 & \text{otherwise}
    \end{cases} \notag \\
    w_3(GR) &= \begin{cases}
        30 & \text{if } GR > 2\%/\text{yr} \\
        15 & \text{if } 1\% < GR \leq 2\% \\
        5 & \text{if } 0 < GR \leq 1\% \\
        0 & \text{otherwise}
    \end{cases} \notag
\end{align}

Final score: $R_{europe} = \min(w_1 + w_2 + w_3, 100)$

\textbf{African Risk Score Formula}:
\begin{align}
    R_{africa} &= 35 + w_1(W_{pc}) + w_2(GR) + w_3(W_{total}) \notag \\
    &\text{Base risk = 35 (no recycling infrastructure)} \notag \\
    w_1(W_{pc}) &= \begin{cases}
        25 & \text{if } W_{pc} > 400 \text{ kg/yr} \\
        15 & \text{if } 300 < W_{pc} \leq 400 \\
        5 & \text{if } 200 < W_{pc} \leq 300 \\
        0 & \text{otherwise}
    \end{cases} \notag \\
    w_2(GR) &= \begin{cases}
        30 & \text{if } GR > 3\%/\text{yr} \\
        20 & \text{if } 2\% < GR \leq 3\% \\
        10 & \text{if } 1\% < GR \leq 2\% \\
        5 & \text{if } 0 < GR \leq 1\% \\
        0 & \text{otherwise}
    \end{cases} \notag \\
    w_3(W_{total}) &= \begin{cases}
        10 & \text{if } W_{total} > 10M \text{ tonnes} \\
        5 & \text{if } 5M < W_{total} \leq 10M \\
        0 & \text{otherwise}
    \end{cases} \notag
\end{align}

Final score: $R_{africa} = \min(35 + w_1 + w_2 + w_3, 100)$

\textbf{Rule-Based vs. ML Approach}:
\begin{itemize}
    \item \textbf{Transparency}: Rules explicitly encode domain expert knowledge
    \item \textbf{Interpretability}: Policy makers can understand exact risk factors
    \item \textbf{Threshold justification}: Based on EU targets (30\%, 50\%) and environmental research
    \item \textbf{Customizability}: Easy to adjust weights based on policy priorities
    \item \textbf{Data efficiency}: Works with limited historical data (no training needed)
\end{itemize}

\textbf{Why Not ML for Risk Scoring?}
\begin{enumerate}
    \item \textbf{Limited labeled data}: No historical "ground truth" risk labels
    \item \textbf{Explainability requirement}: Policy decisions need transparent justification
    \item \textbf{Expert knowledge}: Environmental thresholds well-established in literature
    \item \textbf{Regulatory alignment}: Thresholds match EU Waste Framework Directive targets
\end{enumerate}

\subsubsection{Methodology Comparison Summary}

\begin{table}[H]
\centering
\caption{Forecasting vs. Risk Assessment: Methodology Comparison}
\begin{tabular}{lll}
\toprule
\textbf{Aspect} & \textbf{Waste Forecasting} & \textbf{Risk Assessment} \\
\midrule
Approach & Machine Learning & Rule-Based Expert System \\
Algorithm & ARIMA(1,1,1) + Linear Regression & Weighted Threshold Rules \\
Input Data & Time series (waste values) & Current metrics \\
Output & Future waste (kg/cap) & Risk score (0-100) \\
Training Required & Yes (3-10 years) & No \\
Interpretability & Autoregressive patterns & Explicit rule thresholds \\
Adaptability & Rolling window + fallback & Configurable weights \\
Validation & Historical backtest + convergence & Domain expert review \\
Use Case & Trend prediction & Priority identification \\
Advantages & Captures non-linear patterns & Transparent, policy-aligned \\
\bottomrule
\end{tabular}
\end{table}

\textbf{Key Insight}: The dashboard combines \textbf{ARIMA time series ML for forecasting} (autoregressive data-driven predictions) with \textbf{rule-based scoring for risk} (expert knowledge), leveraging the strengths of both approaches. ARIMA captures waste generation patterns by learning from actual values, not just temporal trends.

\section{Dashboard Features and Screenshots}

\subsection{Dashboard Architecture}

The dashboard is implemented using \textbf{Streamlit} (Python 3.x) with the following technology stack:
\begin{itemize}
    \item \textbf{Data manipulation}: Pandas 2.1.1, NumPy
    \item \textbf{Visualization}: Plotly 5.17.0 (interactive charts)
    \item \textbf{Machine Learning}: scikit-learn 1.5.1
    \item \textbf{Deployment}: Streamlit 1.28.0
\end{itemize}

\subsection{Navigation Structure}

\subsubsection{Region Selection}
Three analysis modes:
\begin{enumerate}
    \item \textbf{Europe (with recycling)}: Comprehensive analysis including recycling rates
    \item \textbf{Africa (generation only)}: Waste production analysis without recycling data
    \item \textbf{North-South Comparison}: Side-by-side regional comparison using waste generation metrics
\end{enumerate}

\subsubsection{Page Sections}

\textbf{For Europe}:
\begin{itemize}
    \item Overview \& KPIs
    \item Temporal Trends
    \item Advanced Analytics
    \item Geographic Analysis
    \item Rankings
    \item Predictions \& Risks
\end{itemize}

\textbf{For Africa}:
\begin{itemize}
    \item Overview \& KPIs
    \item Waste Production
    \item Geographic Analysis
    \item Rankings
    \item Predictions \& Risks
\end{itemize}

\textbf{For North-South Comparison}:
\begin{itemize}
    \item Overview \& KPIs (regional breakdown)
    \item Geographic Analysis (combined world map)
    \item Rankings (comparative rankings)
    \item Predictions \& Risks (dual risk assessment)
\end{itemize}

\subsection{Key Visualizations}

\subsubsection{Overview \& KPIs Dashboard}
\textbf{Purpose}: Provides at-a-glance performance metrics

\textbf{Components}:
\begin{itemize}
    \item Gradient KPI cards with key statistics
    \item Regional averages (Europe vs Africa in comparison mode)
    \item Champion/leader identification
    \item Target achievement metrics (30\% recycling goal)
    \item Comparative bar charts with regional color coding
\end{itemize}

\begin{figure}[H]
    \centering
    \includegraphics[width=0.95\textwidth]{screenshots/01_overview_europe.png}
    \caption{Overview \& KPIs Dashboard - Europe Mode showing recycling rates and waste production}
    \label{fig:overview_europe}
\end{figure}

\begin{figure}[H]
    \centering
    \includegraphics[width=0.95\textwidth]{screenshots/02_overview_comparison.png}
    \caption{Overview \& KPIs Dashboard - North-South Comparison showing regional breakdown}
    \label{fig:overview_comparison}
\end{figure}

\subsubsection{Geographic Choropleth Maps}
\textbf{Purpose}: Spatial visualization of waste metrics

\textbf{Features}:
\begin{itemize}
    \item \textbf{Europe map}: Recycling rate (RdYlGn color scale)
    \item \textbf{Africa map}: Waste per capita (Reds color scale)
    \item \textbf{Combined map}: North-South comparison with natural earth projection
    \item Interactive hover data with country details
    \item Optimized zoom: Latitude [-40°, 75°], Longitude [-25°, 55°]
\end{itemize}

\begin{figure}[H]
    \centering
    \includegraphics[width=0.95\textwidth]{screenshots/03_geographic_europe.png}
    \caption{Geographic Analysis - Europe recycling rate choropleth map}
    \label{fig:geographic_europe}
\end{figure}

\begin{figure}[H]
    \centering
    \includegraphics[width=0.95\textwidth]{screenshots/04_geographic_africa.png}
    \caption{Geographic Analysis - Africa waste per capita distribution}
    \label{fig:geographic_africa}
\end{figure}

\begin{figure}[H]
    \centering
    \includegraphics[width=0.95\textwidth]{screenshots/05_geographic_combined.png}
    \caption{Geographic Analysis - Combined North-South comparison world map}
    \label{fig:geographic_combined}
\end{figure}

\subsubsection{Advanced Analytics}
\textbf{Purpose}: Deep dive into correlations and patterns

\textbf{Visualizations}:
\begin{itemize}
    \item \textbf{Correlation heatmap}: Inter-country recycling rate correlations
    \item \textbf{Trend analysis}: Dual-axis charts (recycling vs waste production)
    \item \textbf{Performance quadrants}: Scatter plot (recycling rate vs waste per capita)
    \item \textbf{Time series decomposition}: Historical trends with annotations
\end{itemize}

\begin{figure}[H]
    \centering
    \includegraphics[width=0.95\textwidth]{screenshots/06_advanced_analytics.png}
    \caption{Advanced Analytics - Correlation heatmap and performance quadrants}
    \label{fig:advanced_analytics}
\end{figure}

\begin{figure}[H]
    \centering
    \includegraphics[width=0.95\textwidth]{screenshots/07_temporal_trends.png}
    \caption{Temporal Trends - Recycling rate evolution over time for selected countries}
    \label{fig:temporal_trends}
\end{figure}

\subsubsection{Predictions \& Forecasting}
\textbf{Purpose}: Future waste generation projections using advanced time series analysis

\textbf{Features}:
\begin{itemize}
    \item ARIMA(1,1,1) model learns from actual waste patterns (autoregressive)
    \item Configurable rolling window (3--10 years)
    \item 5-year forecast horizon
    \item Historical data overlaid with predictions
    \item Visual indicators showing model used (ARIMA vs. Linear Regression fallback)
    \item Country-specific trend lines with pattern recognition
\end{itemize}

\begin{figure}[H]
    \centering
    \includegraphics[width=0.95\textwidth]{screenshots/08_predictions_europe.png}
    \caption{Predictions \& Risks - ML forecasting for European countries}
    \label{fig:predictions_europe}
\end{figure}

\begin{figure}[H]
    \centering
    \includegraphics[width=0.95\textwidth]{screenshots/09_predictions_africa.png}
    \caption{Predictions \& Risks - Waste generation forecasts for African countries}
    \label{fig:predictions_africa}
\end{figure}

\subsubsection{Risk Assessment Dashboard}
\textbf{Purpose}: Environmental risk scoring and prioritization

\textbf{Components}:
\begin{itemize}
    \item Horizontal bar charts with risk scores (0--100)
    \item Color-coded risk levels (green < 40 < yellow < 70 < red)
    \item Risk factor breakdown (recycling, waste PC, growth rate)
    \item Regional comparison (Europe vs Africa)
    \item Warning boxes explaining methodology differences
\end{itemize}

\begin{figure}[H]
    \centering
    \includegraphics[width=0.95\textwidth]{screenshots/10_europ_ranking.png}
    \caption{Rankings - European country performance rankings by recycling rate and waste production}
    \label{fig:rankings}
\end{figure}

\begin{figure}[H]
    \centering
    \includegraphics[width=0.95\textwidth]{screenshots/11_risk_assessment_comparison.png}
    \caption{Risk Assessment - North-South comparative risk analysis}
    \label{fig:risk_comparison}
\end{figure}

\begin{figure}[H]
    \centering
    \includegraphics[width=0.95\textwidth]{screenshots/12_rankings.png}
    \caption{Rankings - Combined country performance rankings (all regions)}
    \label{fig:rankings_all}
\end{figure}

\subsection{User Interaction Features}

\subsubsection{Filters}
\begin{itemize}
    \item \textbf{Country selection}: Multi-select (max 10 countries)
    \item \textbf{Year range}: Sliding selector with available years
    \item \textbf{Region mode}: Toggle between Europe/Africa/Comparison
    \item \textbf{Window size}: Forecasting model configuration
\end{itemize}

\subsubsection{Responsive Design}
\begin{itemize}
    \item Gradient backgrounds for visual appeal
    \item Custom CSS styling for consistency
    \item Adaptive chart heights based on data volume
    \item Insight boxes with contextual information
    \item Mobile-friendly column layouts
\end{itemize}

\subsection{Visualization Design Rationale}

\subsubsection{Color Psychology and Semantic Meaning}
Color choices in this dashboard are carefully selected to convey meaning intuitively:

\textbf{Green color scheme:}
\begin{itemize}
    \item \textbf{Usage}: Recycling rates, positive environmental actions
    \item \textbf{Rationale}: Green universally represents nature, sustainability, and positive environmental outcomes
    \item \textbf{Example}: KPI cards showing recycling rates use green gradients (\#11998e to \#38ef7d)
    \item \textbf{Impact}: Users immediately associate higher values with better performance
\end{itemize}

\textbf{Red color scheme:}
\begin{itemize}
    \item \textbf{Usage}: Waste generation, environmental risks, problems
    \item \textbf{Rationale}: Red signals danger, urgency, and problems requiring attention
    \item \textbf{Example}: Waste per capita metrics use red gradients (\#eb3349 to \#f45c43), choropleth maps showing waste use "Reds" scale
    \item \textbf{Impact}: Darker red intensity = more severe problem (intuitive intensity mapping)
\end{itemize}

\textbf{Blue color scheme:}
\begin{itemize}
    \item \textbf{Usage}: Target achievement, European region identification
    \item \textbf{Rationale}: Blue represents trust, stability, and is associated with EU visual identity
    \item \textbf{Example}: 30\% target achievement KPIs use blue-cyan gradients (\#4facfe to \#00f2fe)
    \item \textbf{Impact}: Conveys professionalism and institutional goals
\end{itemize}

\textbf{Orange color scheme:}
\begin{itemize}
    \item \textbf{Usage}: African region identification, sector-specific data
    \item \textbf{Rationale}: Orange suggests warmth and is culturally neutral
    \item \textbf{Example}: Regional comparisons use orange for Africa (\#f5a742)
    \item \textbf{Impact}: Clear visual distinction from European data
\end{itemize}

\textbf{Purple color scheme:}
\begin{itemize}
    \item \textbf{Usage}: Champions, excellence, achievement
    \item \textbf{Rationale}: Purple historically associated with excellence and leadership
    \item \textbf{Example}: Best performer KPIs use purple gradients (\#667eea to \#764ba2)
    \item \textbf{Impact}: Celebrates top performers with distinctive, prestigious color
\end{itemize}

\textbf{Diverging scales (RdBu):}
\begin{itemize}
    \item \textbf{Usage}: Correlation matrices
    \item \textbf{Rationale}: Blue = positive correlation, red = negative, white = neutral
    \item \textbf{Impact}: Immediately identifies similar vs. opposite patterns
\end{itemize}

\subsubsection{Chart Type Selection Criteria}
Each visualization type is chosen for specific analytical purposes:

\textbf{1. Choropleth Maps (Geographic Analysis)}
\begin{itemize}
    \item \textbf{Best for}: Spatial distribution and geographic patterns
    \item \textbf{Why chosen}: Humans excel at processing geographic information; instantly reveals regional clusters
    \item \textbf{Color mapping}: Sequential scales (Reds, RdYlGn) show intensity gradients
    \item \textbf{Limitation}: Requires country codes; not suitable for time-series
\end{itemize}

\textbf{2. Bar Charts (Rankings, Comparisons)}
\begin{itemize}
    \item \textbf{Best for}: Comparing discrete categories (countries)
    \item \textbf{Why chosen}: Length encoding is highly accurate for human perception
    \item \textbf{Orientation}: Horizontal bars for country names (better readability)
    \item \textbf{Sorting}: Always sorted for easy identification of extremes
\end{itemize}

\textbf{3. Line Charts (Temporal Trends)}
\begin{itemize}
    \item \textbf{Best for}: Tracking changes over time
    \item \textbf{Why chosen}: Continuous lines naturally represent temporal continuity
    \item \textbf{Multiple series}: Different colors for countries enable comparison
    \item \textbf{Markers}: Added to indicate actual data points vs. interpolated
\end{itemize}

\textbf{4. Stacked Area Charts (Sector Analysis)}
\begin{itemize}
    \item \textbf{Best for}: Part-to-whole relationships over time
    \item \textbf{Why chosen}: Shows both individual sector contributions and total volume simultaneously
    \item \textbf{Color strategy}: Warm color palette (red-orange-yellow) for waste sectors
    \item \textbf{Interpretation}: Band width = sector contribution; total height = aggregate
\end{itemize}

\textbf{5. Scatter Plots with Quadrants (Performance Analysis)}
\begin{itemize}
    \item \textbf{Best for}: Bivariate relationships and categorization
    \item \textbf{Why chosen}: Reveals correlation between two metrics (recycling vs. waste production)
    \item \textbf{Quadrant division}: Median-based splits create 4 performance categories
    \item \textbf{Color coding}: 4 distinct colors for each quadrant (green=best, red=worst)
\end{itemize}

\textbf{6. Heatmaps (Correlation Matrices)}
\begin{itemize}
    \item \textbf{Best for}: Multivariate relationships in matrix form
    \item \textbf{Why chosen}: Efficiently displays $n \times n$ country correlations
    \item \textbf{Color scale}: Diverging (RdBu) centered at zero correlation
    \item \textbf{Use case}: Identify countries with similar policy trajectories
\end{itemize}

\textbf{7. KPI Cards with Gradients}
\begin{itemize}
    \item \textbf{Best for}: At-a-glance key metrics
    \item \textbf{Why chosen}: Large numbers with semantic colors for immediate comprehension
    \item \textbf{Design}: Gradient backgrounds prevent monotony while maintaining professionalism
    \item \textbf{Hierarchy}: Font sizes (3rem for values) ensure primary information dominates
\end{itemize}

\subsubsection{Accessibility and Usability Considerations}

\textbf{Color blindness accommodation:}
\begin{itemize}
    \item Diverging scales (RdBu) work for most color vision deficiencies
    \item Text labels and values always accompany color coding
    \item Hover interactions provide detailed numeric data
    \item High contrast ratios between text and backgrounds (WCAG AA compliant)
\end{itemize}

\textbf{Cognitive load reduction:}
\begin{itemize}
    \item Maximum 10 countries per selection (prevents chart clutter)
    \item Insight boxes explain complex visualizations
    \item Consistent color mapping across all pages
    \item Progressive disclosure: simple overview → detailed analytics
\end{itemize}

\textbf{Interactive elements:}
\begin{itemize}
    \item Hover tooltips for detailed information on demand
    \item Zoomable maps for geographic analysis
    \item Sortable tables with gradient backgrounds
    \item Filter controls in sidebar for easy access
\end{itemize}

\textbf{Responsive design:}
\begin{itemize}
    \item Adaptive chart heights based on data volume (e.g., $\text{height} = \max(400, n_{countries} \times 30)$)
    \item Column layouts that stack on smaller screens
    \item Consistent 16:9 aspect ratios for screenshots and presentations
\end{itemize}

\subsubsection{Evidence-Based Design Decisions}

\textbf{Why NOT use:}
\begin{itemize}
    \item \textbf{Pie charts}: Poor for precise comparisons; replaced with bar charts
    \item \textbf{3D visualizations}: Add distortion without information; all charts are 2D
    \item \textbf{Excessive animation}: Distracting; only subtle hover effects used
    \item \textbf{Rainbow color scales}: Misleading perceptual uniformity; use ColorBrewer scales
\end{itemize}

\textbf{Design principles followed:}
\begin{enumerate}
    \item \textbf{Data-ink ratio maximization} (Tufte): Remove unnecessary chart junk
    \item \textbf{Pre-attentive processing}: Use color/size to guide attention
    \item \textbf{Gestalt principles}: Group related information spatially
    \item \textbf{Consistency}: Same metrics always use same colors across dashboard
    \item \textbf{Context provision}: Every chart has clear title, labels, and interpretation guide
\end{enumerate}

\section{Results Interpretation}

\subsection{Dashboard Development Journey}

This dashboard underwent significant enhancements to ensure visualizations are intuitive and scientifically rigorous:

\textbf{Version 2.0 Improvements} (November 2025):
\begin{enumerate}
    \item \textbf{Expanded Coverage}: Increased from 31 to 49 countries (+58\%)
    \begin{itemize}
        \item European countries: 21 → 27 (added Estonia, Hungary, Iceland, Latvia, Lithuania, Turkey)
        \item African countries: 10 → 22 (added 12 countries for better continental representation)
    \end{itemize}
    
    \item \textbf{New Visualization}: Stacked area chart for waste by sector
    \begin{itemize}
        \item Displays 4 economic sectors (households, construction, manufacturing, services)
        \item Semantic color palette (warm reds/oranges for waste intensity)
        \item Enables identification of sector-specific growth patterns
    \end{itemize}
    
    \item \textbf{Semantic Color Redesign}: Colors now convey meaning instantly
    \begin{itemize}
        \item Green gradients = Recycling/positive actions
        \item Red gradients = Waste/risks/problems
        \item Blue = Targets/achievements (Europe theme)
        \item Orange = Africa regional identity
        \item Purple = Excellence/champions
    \end{itemize}
    
    \item \textbf{Explanation Boxes}: Added "Why This Visualization?" panels on every page
    \begin{itemize}
        \item Justifies chart type selection with visualization science
        \item Explains color psychology rationale
        \item Provides interpretation guides for complex charts
        \item Cites evidence-based design principles (Tufte, Cleveland, ColorBrewer)
    \end{itemize}
    
    \item \textbf{Accessibility Enhancements}: WCAG 2.1 AA compliance
    \begin{itemize}
        \item Colorblind-friendly diverging scales (RdBu for correlations)
        \item Text labels always accompany color encoding
        \item High contrast ratios (minimum 4.5:1)
        \item Maximum 10 country selection to reduce cognitive load
    \end{itemize}
\end{enumerate}

\subsection{European Performance Analysis}

\subsubsection{Recycling Leaders}
\textbf{Top performers} (based on latest available data):
\begin{itemize}
    \item \textbf{Germany}: Consistently above 45\% recycling rate
    \item \textbf{Austria}: Strong upward trend, approaching 30\% target
    \item \textbf{Belgium}: Stable performance around 35\%
\end{itemize}

\textbf{Key observations}:
\begin{itemize}
    \item Northern European countries show higher recycling rates
    \item Correlation between GDP per capita and recycling infrastructure
    \item Policy effectiveness: Countries with deposit-return systems perform better
\end{itemize}

\subsubsection{Areas for Improvement}
\textbf{Underperforming countries}:
\begin{itemize}
    \item Several Eastern European countries below 20\% recycling rate
    \item Wide disparity within EU region (range: 10\%--50\%)
    \item Slower progress in countries with recent EU membership
\end{itemize}

\subsubsection{Waste Generation Trends}
\textbf{Patterns observed}:
\begin{itemize}
    \item Average European waste: 450--550 kg/capita/year
    \item Slight decline in some countries post-2010 (economic crisis impact)
    \item Decoupling of waste generation from economic growth in Nordic countries
\end{itemize}

\subsection{African Waste Management Landscape}

\subsubsection{Generation Patterns}
\textbf{Observations}:
\begin{itemize}
    \item High variability: 100--400 kg/capita/year
    \item \textbf{South Africa}: Highest waste per capita ($\sim$400 kg/year)
    \item \textbf{Kenya, Benin}: Lower rates (150--200 kg/year) but rapidly growing
    \item Urban-rural divide significant (data reflects primarily urban areas)
\end{itemize}

\subsubsection{Growth Dynamics}
\textbf{Alarming trends}:
\begin{itemize}
    \item Growth rates exceeding 3--5\%/year in several countries
    \item Rapid urbanization driving waste increase
    \item Limited infrastructure development to match growing volumes
    \item Informal waste sector handles 50--80\% of waste in some cities
\end{itemize}

\subsubsection{Data Limitations}
\begin{itemize}
    \item Sparse data points for several countries (Kenya, Benin)
    \item Interpolation necessary but introduces uncertainty
    \item No recycling data available (informal recycling not captured)
    \item Collection systems incomplete in rural areas
\end{itemize}

\subsection{North-South Comparison}

\subsubsection{Key Disparities}
\begin{center}
\begin{tabular}{lcc}
\toprule
\textbf{Metric} & \textbf{Europe} & \textbf{Africa} \\
\midrule
Avg. Recycling Rate & 25--35\% & No data \\
Avg. Waste/Capita & 500 kg/yr & 250 kg/yr \\
Infrastructure & Advanced & Limited \\
Growth Rate & Stable (1--2\%) & Rapid (3--5\%) \\
\bottomrule
\end{tabular}
\end{center}

\subsubsection{Interpretation}
\begin{enumerate}
    \item \textbf{Consumption patterns}: European waste generation reflects higher consumption levels
    \item \textbf{Infrastructure gap}: Africa lacks formal recycling systems present in Europe
    \item \textbf{Future challenge}: Africa's rapid growth suggests future crisis without intervention
    \item \textbf{Opportunity}: Africa can leapfrog to circular economy models with proper investment
\end{enumerate}

\subsection{Predictive Insights}

\subsubsection{5-Year Forecasts}
\textbf{Europe}:
\begin{itemize}
    \item Stable or slight decline in waste generation expected
    \item Recycling rates projected to continue gradual increase
    \item EU circular economy package may accelerate improvements
\end{itemize}

\textbf{Africa}:
\begin{itemize}
    \item Continued rapid increase in waste volumes predicted
    \item Without intervention, waste could double in some countries by 2030
    \item Urban areas will face severe capacity challenges
\end{itemize}

\subsubsection{Risk Assessment Results}
\textbf{Highest risk countries}:
\begin{itemize}
    \item \textbf{Europe}: Countries with $<$20\% recycling and high waste/capita
    \item \textbf{Africa}: Large urban centers (Egypt, South Africa) with rapid growth
\end{itemize}

\textbf{Risk factors}:
\begin{itemize}
    \item Combination of high generation + low infrastructure = critical risk
    \item Countries in transition phase most vulnerable
    \item Climate change compounding waste management challenges
\end{itemize}

\section{Recommendations and Action Plans}

\subsection{For European Countries}

\subsubsection{Short-term Actions (1--2 years)}
\begin{enumerate}
    \item \textbf{Harmonize data collection}: Ensure annual reporting instead of biennial
    \item \textbf{Peer learning programs}: Connect low-performing with high-performing countries
    \item \textbf{Economic incentives}: Expand deposit-return schemes to more product categories
    \item \textbf{Public awareness}: Launch targeted campaigns in underperforming regions
\end{enumerate}

\subsubsection{Long-term Strategies (5--10 years)}
\begin{enumerate}
    \item \textbf{Circular economy transition}: Move from recycling focus to waste prevention
    \item \textbf{Extended Producer Responsibility}: Strengthen EPR frameworks
    \item \textbf{Digital tracking}: Implement blockchain-based waste tracking systems
    \item \textbf{Innovation funding}: Support clean-tech startups in waste sector
\end{enumerate}

\subsection{For African Countries}

\subsubsection{Immediate Priorities (1--3 years)}
\begin{enumerate}
    \item \textbf{Infrastructure investment}: Establish basic collection and sorting facilities
    \item \textbf{Formalize informal sector}: Integrate waste pickers into formal systems
    \item \textbf{Data collection}: Implement systematic waste measurement programs
    \item \textbf{Regional cooperation}: Share resources and expertise across borders
\end{enumerate}

\subsubsection{Medium-term Goals (3--7 years)}
\begin{enumerate}
    \item \textbf{Recycling infrastructure}: Build modern material recovery facilities
    \item \textbf{Policy frameworks}: Develop comprehensive waste management legislation
    \item \textbf{Technology transfer}: Adapt proven European technologies to African context
    \item \textbf{Capacity building}: Train workforce in waste management professions
\end{enumerate}

\subsubsection{Strategic Vision (10+ years)}
\begin{enumerate}
    \item \textbf{Leapfrog strategy}: Skip landfill-heavy models, move directly to circular economy
    \item \textbf{Green jobs}: Create employment through waste-to-resource industries
    \item \textbf{Regional hubs}: Establish specialized recycling centers serving multiple countries
    \item \textbf{Innovation ecosystems}: Foster African solutions for African challenges
\end{enumerate}

\subsection{Global Cooperation Initiatives}

\subsubsection{North-South Technology Transfer}
\begin{itemize}
    \item \textbf{Knowledge exchange programs}: Secondment of experts
    \item \textbf{Open-source platforms}: Share dashboard and analytics tools
    \item \textbf{Affordable technology}: Adapt European innovations for African budgets
    \item \textbf{Joint research}: Collaborative studies on waste management
\end{itemize}

\subsubsection{Financing Mechanisms}
\begin{itemize}
    \item \textbf{Green bonds}: International funding for waste infrastructure
    \item \textbf{Carbon credits}: Link waste reduction to climate finance
    \item \textbf{Public-private partnerships}: Leverage private sector efficiency
    \item \textbf{Multilateral support}: World Bank, African Development Bank involvement
\end{itemize}

\subsection{Dashboard Enhancement Roadmap}

\subsubsection{Data Expansion}
\begin{enumerate}
    \item Include more African countries (target: 20+ countries)
    \item Add waste composition data (plastic, organic, electronic, etc.)
    \item Integrate informal sector estimates
    \item Real-time data feeds where available
\end{enumerate}

\subsubsection{Advanced Analytics}
\begin{enumerate}
    \item \textbf{Enhanced ARIMA models}: Seasonal ARIMA (SARIMA) for cyclic patterns
    \item \textbf{Deep learning models}: LSTM networks for long-term forecasting
    \item \textbf{Scenario analysis}: "What-if" policy impact simulations
    \item \textbf{Optimization algorithms}: Route optimization for waste collection
    \item \textbf{Causal inference}: Identify policy effectiveness rigorously
\end{enumerate}

\subsubsection{User Experience}
\begin{enumerate}
    \item Multi-language support (French, Arabic, Swahili)
    \item Mobile application version
    \item Automated report generation (PDF exports)
    \item API for third-party integrations
\end{enumerate}

\subsection{Policy Recommendations}

\subsubsection{For International Organizations (UN, AU, EU)}
\begin{itemize}
    \item Mandate standardized waste reporting across all nations
    \item Establish global recycling targets aligned with SDGs
    \item Create waste management technology transfer fund
    \item Support regional waste management compacts
\end{itemize}

\subsubsection{For National Governments}
\begin{itemize}
    \item Adopt extended producer responsibility laws
    \item Invest in waste-to-energy where appropriate
    \item Ban single-use plastics with transition support
    \item Integrate waste management into climate adaptation plans
\end{itemize}

\subsubsection{For Municipal Authorities}
\begin{itemize}
    \item Implement pay-as-you-throw pricing systems
    \item Expand separate collection for recyclables
    \item Partner with social enterprises for waste collection
    \item Use this dashboard for data-driven planning
\end{itemize}

\section{Conclusion}

This environmental dashboard represents a comprehensive tool for understanding and addressing waste management challenges in Europe and Africa. Key takeaways include:

\begin{enumerate}
    \item \textbf{Data reveals stark disparities}: Europe's advanced recycling infrastructure contrasts sharply with Africa's nascent systems
    \item \textbf{Predictive power}: Machine learning forecasts enable proactive rather than reactive policy
    \item \textbf{Actionable insights}: Risk assessment identifies priority areas for intervention
    \item \textbf{Comparative framework}: North-South analysis highlights opportunities for collaboration and learning
\end{enumerate}

The dashboard's integration of multiple data sources, interpolation techniques, and predictive models provides a robust foundation for evidence-based environmental decision-making. As waste volumes continue to grow globally, tools like this become essential for tracking progress, identifying challenges, and guiding sustainable solutions.

\textbf{Future work} should focus on expanding geographic coverage, incorporating real-time data streams, and developing more sophisticated predictive models. The ultimate goal is to support the transition to circular economies where waste is minimized, resources are conserved, and environmental health is prioritized.

\vspace{1cm}

\noindent\textbf{Dashboard Access}: The interactive dashboard is deployed using Streamlit and can be accessed at the project repository or hosted instance.

\vspace{0.5cm}

\noindent\textbf{Technical Documentation}: Complete source code, data preprocessing scripts, and API documentation are available in the project GitHub repository.

\section*{References}

\begin{enumerate}
    \item \textbf{Our World in Data - Total Waste Generation}. \textit{Global waste generation statistics (2000-2021)}. \\
    \url{https://ourworldindata.org/grapher/total-waste-generation?time=earliest..2021}
    
    \item \textbf{Our World in Data - Recycling Rates}. \textit{Municipal waste recycling rates by country (1990-2015)}. \\
    \url{https://ourworldindata.org/grapher/recycling-rates-paper-and-cardboard?tab=line}
    
    \item \textbf{OECD Environment Statistics}. \textit{Municipal Waste, Generation and Treatment}. Organisation for Economic Co-operation and Development, 2024.
    
    \item \textbf{UN Environment Programme}. \textit{Global Waste Management Outlook}. United Nations Environment Programme, 2023.
    
    \item \textbf{World Bank Open Data}. \textit{Population estimates by country}. The World Bank Group, 2024.
    
    \item Box, G. E. P., Jenkins, G. M., Reinsel, G. C., \& Ljung, G. M. (2015). \textit{Time Series Analysis: Forecasting and Control} (5th ed.). Wiley.
    
    \item Cleveland, W. S., \& McGill, R. (1984). \textit{Graphical Perception: Theory, Experimentation, and Application to the Development of Graphical Methods}. Journal of the American Statistical Association, 79(387), 531-554.
    
    \item Tufte, E. R. (2001). \textit{The Visual Display of Quantitative Information} (2nd ed.). Graphics Press.
    
    \item European Commission (2018). \textit{EU Waste Framework Directive 2008/98/EC}. Official Journal of the European Union.
    
    \item ColorBrewer. \textit{Color advice for cartography}. Cynthia Brewer, Pennsylvania State University. \\
    \url{https://colorbrewer2.org}
\end{enumerate}

\end{document}
